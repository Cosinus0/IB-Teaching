\documentclass[10pt,a4paper]{article}
\usepackage[utf8x]{inputenc}
\usepackage[T1]{fontenc}
\usepackage{ucs}
\usepackage[english]{babel}
\usepackage{amsmath}
\usepackage{amsfonts}
\usepackage{amssymb}
\usepackage{graphicx}
\usepackage{float}
\usepackage{tikz}
\usepackage{venndiagram}
\graphicspath{{./figures/}}
\begin{document}
	\title{Measurement}
	\author{Nathan Hugh Barr}
	\maketitle
	
	Weeks: $1$\\
	Dates: Monday $20/1$, Wednesday $22/1$ and Thursday $23/1$
	
	\section*{Monday 20/1}
	\textbf{First half module:} Similar shapes and solids
	\begin{itemize}
		\item Recap of enlargement factor
		\item Enlargement of length, area and volume
		\item Questions 
	\end{itemize}

	\textbf{Second half module:} Construction of 3d objects
	\begin{itemize}
		\item Making cubes and Pyramids
		\item Coordinates in 3-D and midpoint formula
		\item Questions
	\end{itemize}

	\section*{Wednesday 22/1}
	\textbf{First half module:} Volume and surface of pyramids, cones and spheres
	\begin{itemize}
		\item Description of pyramids and cones
		\item Formulas
		\item Questions
	\end{itemize}

	\textbf{Second half module:} Continuation
	
	\section*{Thursday 23/1}
	\textbf{First half module:} Converting between units
	\begin{itemize}
		\item Examples
		\item Questions
	\end{itemize}

	\textbf{Second half module:} Continuation

\section{Notes to Monday 20/1}
Ask these questions:

\begin{itemize}
	\item What is enlargement?
	\item What is the scale factor?
	\item In what intervals is there enlargement, reduction, inverted enlargement and reduction.
\end{itemize}

How do we use the scale factor to enlarge or reduce, length, area and volume?

Enlargement factor $n$
\begin{align}
\text{Length} 1:n \\
\text{Area} 1:n^2 \\
\text{Volume} 1:n^3 
\end{align}

\begin{figure}[H]
	\includegraphics[width=\textwidth]{ex1.png}
\end{figure}

\begin{figure}[H]
	\includegraphics[width=\textwidth]{ex2.png}
\end{figure}


\textbf{Questions:}
\begin{itemize}
	\item 4,5,9 Page(281-282)
\end{itemize}


Activity: Constructing 3D shapes - Cubes and Triangles

\begin{figure}[H]
	\includegraphics[width=\textwidth]{ex2a.png}
	\end{figure}


\textbf{Coordinates in 3D}
\begin{figure}[H]
	\includegraphics[width=\textwidth]{ex3.png}
\end{figure}

The midpoint of a line formula:

\begin{equation}
\left( \frac{a+p}{2}, \frac{b+q}{2}, \frac{c+r}{2}  \right)
\end{equation}

\textbf{Questions:}
\begin{itemize}
	\item 18.2 - All questions Page(341-342)
\end{itemize}

\section{Notes to Wednesday 22/1}

A pyramid or a cone has a base and an apex. Rays are drawn from the edge of the base and converges at a point, the apex. Pyramids can have square or triangle bases, and cones have a circular base.\\

The volume of a pyramid is given:

\begin{equation}
\text{Volume} = \frac{1}{3} \times \text{area of the base} \times \text{Perpendicular height}
\end{equation}

The volume of a cone is given:

\begin{equation}
\text{Volume} = \frac{1}{3} \times \pi r^2 h
\end{equation}
where $r$ is the radius of the base, and $h$ is the perpendicular height. And the surface area:

\begin{equation}
A = \pi r l
\end{equation}
where l is the slant length.


The volume for a sphere is given:

\begin{equation}
V = \frac{4}{3} \times \pi r^3
\end{equation}

and the surface area:

\begin{equation}
A = 4\pi r^2
\end{equation}

\textbf{Examples:}

\begin{figure}[H]
	\includegraphics[width=\textwidth]{ex4.png}
\end{figure}

Ex - A spherical ball bearing has a diameter of $6$ mm. Find its volume?

\begin{figure}[H]
	\includegraphics[width=\textwidth]{ex5.png}
\end{figure}


\textbf{Questions:}
\begin{itemize}
	\item 18.3 - All questions Page(345-346)
\end{itemize}


\section{Notes to Thursday 23/1}
Converting units:

\begin{align}
1 \text{cm} &= 10 \text{mm}\\
1 \text{cm}^2 &= 10 \times 10 = 100 \text{mm}^2\\
1 \text{cm}^3 &= 10 \times 10 \times 10 = 100 \text{mm}^3\\
1 \text{km} = 1000 \text{meters}
\end{align}

\textbf{Questions:}
\begin{itemize}
\item 18.4 1-10 Pages 347
\end{itemize}

\end{document}