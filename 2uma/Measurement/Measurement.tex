\documentclass[10pt,a4paper]{article}
\usepackage[utf8x]{inputenc}
\usepackage[T1]{fontenc}
\usepackage{ucs}
\usepackage[english]{babel}
\usepackage{amsmath}
\usepackage{amsfonts}
\usepackage{amssymb}
\usepackage{graphicx}
\usepackage{float}
\usepackage{tikz}
\usepackage{venndiagram}
\graphicspath{{./figures/}}
\begin{document}
	\title{Measurement}
	\author{Nathan Hugh Barr}
	\maketitle
	
	Weeks: $1$\\
	Dates: Monday $20/1$ and Tuesday $21/1$
	
	\section*{Monday the 20/1}
	
	\textbf{First Half Module}: Probability
	\begin{itemize}
		\item Making predictions using probability
		\item Birthday paradox
		\item Questions
	\end{itemize}

	\textbf{Second Half Module}:Measurement
	\begin{itemize}
		\item Circles, Arcs and Sectors
		\item Surface area
	\end{itemize}

	\section*{Tuesday the 21/1}
	\textbf{First Half Module}:Measurement
	\begin{itemize}
		\item Volume
		\item Questions
	\end{itemize}

	\textbf{Second Half Module}
	\begin{itemize}
		\item Capacity
		\item Questions
	\end{itemize}

\section{Notes to Monday 20/1}

As we have seen previously, the experimental probability is equal to the relative frequency of the event.

\begin{align}
\text{Experimental probability} &= \text{relative frequency of the event}\\
&= \frac{\text{number of times event occurs}}{\text{number of trials}}
\end{align}

If we rearrange this equations we get:

\begin{equation}
\text{number of times event occurs} = \text{experimental probability} \times \text{number of trials}
\end{equation}

So before we were starting with an experiment and using the outcome to say something about the probability. This equation is the opposite! We use a theoretical probability to predict the results.

\textbf{Definition:} If there are $n$ trials of an experiment, and an event has probability $p$ of occurring in each of the trials, then the number of times we expect the event to occur is $np$.

\begin{figure}[H]
	\includegraphics[width=\textwidth]{Mjordan_example.png}
\end{figure}

$np$ will not likely be an integer, what do we do? And is this a problem?\\

\textbf{The birthday paradox}


\textbf{Questions:}
\begin{itemize}
	\item 10J - 1,3,5 Page(275-276)
\end{itemize}

\textbf{Measurement - Circles, arcs and sectors}
We will be revising measurement, mostly area and volume of different shapes. The first shape we will be looking at is the circle.

\begin{figure}[H]
	\includegraphics[width = \textwidth]{circle_meas.png}
\end{figure}

\begin{figure}[H]
	\includegraphics[width = \textwidth]{circle_ex.png}
\end{figure}

\textbf{Surface Area}
The surface area of a three-dimensional figure with plane faces is the sum of the areas of the faces.

\begin{figure}[H]
	\includegraphics[width = \textwidth]{Sa_ex.png}
\end{figure}

\textbf{Solids with curved surfaces}
We have just looked at flat surfaces, but some solids have curved surfaces, these can be calculated.

\begin{figure}[H]
	\includegraphics[width=\textwidth]{Sa_curved.png}
\end{figure}

\begin{figure}[H]
	\includegraphics[width=\textwidth]{Sa_curvedex1.png}
\end{figure}

\begin{figure}[H]
	\includegraphics[width=\textwidth]{Sa_curvedex2.png}
\end{figure}

\textbf{Questions:}
\begin{itemize}
	\item 6A Odd questions Page(147-148)
	\item 6B.1 1,2,3,5 Page(149-150)
	\item 6B.2 1,3,5,7,9 Page(152-153) 
\end{itemize}


\section{Notes to Tuesday 21/1}

\textbf{Volume definition:} The volume of a solid is the amount of space it occupies.

\begin{equation}
\text{Volume} = \text{area of cross-section} \times \text{length}
\end{equation}

\begin{figure}[H]
	\includegraphics[width=\textwidth]{vol1.png}
\end{figure}

\begin{figure}[H]
	\includegraphics[width=\textwidth]{vol2.png}
\end{figure}

\begin{figure}[H]
	\includegraphics[width=\textwidth]{vol3.png}
\end{figure}

\textbf{Tapered solids}
For solids such as pyramids and cones, tapered solids, the following equation is used.

\begin{equation}
\text{Volume} = \frac{1}{3}(\text{area of base} \times \text{height})
\end{equation}

For spheres the volume is given

\begin{equation}
\text{Volume} = \frac{4}{3} \pi r^3
\end{equation}

\begin{figure}[H]
	\includegraphics[width=\textwidth]{vol4.png}
\end{figure}

\textbf{Questions:}
\begin{itemize}
	\item 6C.1  1,2,4,7 Page(155-156)
	\item 6C.2  1,5,6  Page(160-161)
\end{itemize}

\section{Capacity}

\textbf{Capactiy:} The capacity of a container is the quantity of fluid it is capable of holding. The capacity belongs to the container, not to the fluid it's self.\\

The definition for $1$mL of water occupies $1$cm$^3$ of space is a cube with $1$cm sides.

\begin{figure}[H]
	\includegraphics[width=0.2\textwidth]{cap1.png}
\end{figure}

\begin{figure}[H]
	\includegraphics[width=0.6\textwidth]{cap2.png}
\end{figure}

\begin{figure}[H]
	\includegraphics[width=0.9\textwidth]{cap3.png}
\end{figure}

\begin{figure}[H]
	\includegraphics[width=\textwidth]{cap4.png}
\end{figure}


\end{document}