\documentclass[10pt,a4paper]{article}
\usepackage[utf8x]{inputenc}
\usepackage[T1]{fontenc}
\usepackage{ucs}
\usepackage[english]{babel}
\usepackage{amsmath}
\usepackage{amsfonts}
\usepackage{amssymb}
\usepackage{graphicx}
\usepackage{float}
\usepackage{tikz}
\usepackage{siunitx}
\usepackage{gensymb}
\usepackage{venndiagram}
\graphicspath{{./figures/}}
\begin{document}

\title{Trigonometry}
\author{Nathan Hugh Barr}
\maketitle

Weeks: $2$

Dates: Monday $6/1$, Wednesday $8/1$, Thursday $9/1$, Tuesday $14/1$, Thursday $16/1$, Friday $17/1$

\section*{Monday 6/1}
\textbf{1st half module:}
\begin{itemize}
	\item Exam - Anu
\end{itemize}

\textbf{2nd half module:}
\begin{itemize}
	\item Math Activity
\end{itemize}

\section*{Wednesday 8/1}
\textbf{1st half module:}

\textbf{Trigonometrical functions}
\begin{itemize}
	\item Sine Ratio
	\item Cosine Ratio
	\item Tangent Ratio
\end{itemize}

\textbf{2nd half module:}

\textbf{Choosing the right trig function?}
\begin{itemize}
	\item SohCahToa
	\item How to use the SohCahToa triangles.
	\item Questions
\end{itemize}

\section*{Thursday 9/1}
\textbf{1st half module:}

\textbf{Recap of the trigonometrical functions}\\

\textbf{Finding an unknown angle}
\begin{itemize}
	\item Inverse functions
	\item Questions
\end{itemize}
\textbf{Angles of elevation and depression}
\begin{itemize}
	\item Elevation
	\item Depression
	\item Questions
\end{itemize}

\textbf{2nd half module:}

\textbf{Multi-stage problems}
	\begin{itemize}
		\item Triangles split into two right angle triangles.
		\item Pythagoras' Theorem
		\item Bearings
		\item Questions
	\end{itemize} 

\section*{Tuesday 14/1}
\textbf{1st half module:}

\textbf{Sine Rule}
\begin{itemize}
	\item Ambiguous case of the sine rule
	\item Questions
\end{itemize}

\textbf{2nd half module:}

\textbf{Cosine Rule}
\begin{itemize}
	\item Questions
\end{itemize}

\section*{Thursday 16/1}
\textbf{1st half module:}

\textbf{Recap of Right Angles}

\textbf{Recap of Sine rule and Cosine rule}\\

\textbf{2nd half module:}

\textbf{Area of a Triangle}
\begin{itemize}
	\item Questions
\end{itemize}

\section*{Friday 17/1}
\textbf{Trigonometry in 3-D:}

\begin{itemize}
	\item Questions
\end{itemize}

\section{Notes Monday 6/1}


\section{Notes Wednesday 8/1}
\textbf{The Sine Ratio:}

Consider two triangles with similar angles. The ratio will be same regardless of the different lengths. The larger triangle has been enlarged by a scale factor. 

\begin{figure}[H]
	\includegraphics[width=\textwidth]{sinefig.png}
\end{figure}

In right angle triangles, the answer obtained by dividing the length opposite the angle by the hypotenuse is called the sine of that angle.

\begin{equation}
\sin(\theta) = \frac{\text{opposite}}{\text{hypotenuse}}
\end{equation}

Remember - Calculators set to degrees (DEG)

To calculate a missing length, Isolate the missing length from the sine formula.


\begin{figure}[H]
	\includegraphics[width=\textwidth]{basicTriangle.png}
\end{figure}
\begin{align}
\sin(\theta) &= \frac{\text{opposite}}{\text{hypotenuse}}\\
\implies \text{opposite} &= \sin(\theta) \times \text{hypotenuse}
\end{align}

\textbf{Example:}
\begin{figure}[H]
	\includegraphics[width=\textwidth]{sineEx.png}
\end{figure}

\begin{align}
\text{opposite} &= \text{hypotenuse} \times \sin(\theta) \\
x &= 4 \times \sin(38\degree)\\
x &= 2.462645901\degree \approx 2.46\degree
\end{align}

Show that the hypotenuse can also be isolated.

\textbf{Cosine Ratio:}

Instead of trying to find the opposite length, the adjacent length can be found using the cosine ratio.

\begin{align}
\cos(\theta) &= \frac{\text{adjacent}}{\text{hypotenuse}}\\
\implies \text{adjacent} &= \text{hypotenuse} \times \cos(\theta)
\end{align}

\begin{figure}[H]
	\includegraphics[width=\textwidth]{Excos.png}
\end{figure}

\textbf{Tangent Ratio:}

The last ratio is defined as:

\begin{align}
\tan(\theta) &= \frac{\text{opposite}}{\text{adjacent}}\\
\implies \text{opposite} &= \text{adjacent} \times \tan(\theta)
\end{align}

\begin{figure}[H]
	\includegraphics[width=\textwidth]{Extan.png}
\end{figure}

\textbf{SohCahToa}
When solving questions the trigonometrical ratio rhyme can be used and triangles can be drawn to help isolate the missing length or angle(but more on that later).

\begin{figure}[H]
	\includegraphics[width=\textwidth]{sohcahtoa.png}
\end{figure} 
\textbf{How to use:} Cover the quantity you are looking for and the triangle will tell to either multiple or divide the quantities you were given in the question.

\textbf{Questions:}
\begin{itemize}
	\item 1 through to 8 (Page 317)
\end{itemize}

\section{Notes Thursday 9/1:}

\textbf{Finding an unknown angle:} 
\begin{itemize}
	\item Inverse - "Reverse" function
	\item Inverse Angle - Calculators -> shift, sin/cos/tan.
	\item \includegraphics[width=\textwidth]{example1.png}
	\item $\tan(\theta)=\frac{3}{4} \implies \theta = \arctan(\frac{3}{4})=36.86989765 \degree \approx 36.9\degree$ 
\end{itemize}

\textbf{Questions:}
\begin{itemize}
	\item 2,4,6,8,10,12 (Page 319)
\end{itemize}

\textbf{Angles of elevation and depression:}

Angle of elevation - Above the horizontal line

Angle of depression - Below the horizontal line.

These problems are a combination of pythagoras' theorem, trigonometrical equations and alternate angles.

\textbf{Example:} A climber is sitting on the summit S of a mountain. He looks down and sees his camp C in the valley below. The direct distance from the summit to the camp is 2200 metres. The summit is at an altitude 420 meters higher than the camp.\\
a: Calculate the angle of elevation of the summit as seen from the camp. Give your answer to the nearest $0.1$\textdegree.

\begin{align}
\sin(\theta) &= \frac{\text{opposite}}{\text{hypotenuse}}\\
\sin(\theta) &= \frac{420}{2200}\\
\theta &= \arcsin(\frac{420}{2200}) = 11.005\degree \approx 11\degree\\
\end{align}

b: Write down the angle of depression of the camp as seen from the summit. $=11\degree$

\textbf{Questions:}
\begin{itemize}
	\item 1,3,4 (Page. 325-326)
\end{itemize}

\textbf{Multi-stage problems:}
Some problems will involve non-right angle triangles, such as:

\begin{figure}[H]
	\includegraphics[width=\textwidth]{example2.png}
\end{figure}

Now you might be able to see, two right angle triangles!

a: Calculate the height BC, to three significant numbers.

b: Calculate length BD.

c: Calculate $\theta$.

\begin{align}
&c^2 = a^2 + b^2\\
\implies &b^2 = c^2 - a^2\\
&BC^2 = 25^2-21^2 = 184\\
\implies &BC = \sqrt{184} = 13.5646 \approx 13.6
\end{align}


\textbf{Use Soh triangle!}	
\begin{align}
\sin(65\degree) &= \frac{BC}{BD} = \frac{13.6}{BD}\\
BD &= \frac{13.56465997}{\sin(65\degree)} = 14.96694629 \approx 15.0 
\end{align}

\begin{align}
\cos(\theta) &= \frac{21}{25}\\
\theta &= \arccos(\frac{21}{25}) = 32.85988038 \approx 32.9 \degree
\end{align}

\textbf{Bearing:}Absolute bearing refers to the angle between the magnetic North (magnetic bearing) or true North (true bearing) and an object.

\textbf{Example:} A ship leaves its harbour and sails due south for 10km. It then sails due East for 20km then stops.
a:How far is the ship from its harbour?

b:The ship wishes to return directly to its habour. On what bearing must it sail?

\begin{align}
d^2 &= 20^2 + 10^2 = 500\\
d &= \sqrt{500} = 22.36067978 \approx 22.4 km
\end{align}

\begin{align}
\tan(\theta) &= \frac{10}{20} = 0.5\\
\theta &= \arctan(0.5) = 26.56505118 \approx 26.6\degree\\
\text{Bearing} &= 270\degree + 26.6\degree = 296.6\degree
\end{align}

\section{Notes Tuesday 14/1}

\textbf{The sine rule:}
Earlier we worked on right angle triangle, and looked at the trigonometrical formulae we can use to solve those problems. But what happens when the triangles are not right angle triangles? We can break them into two right angle triangles or we can use the Sine rule, and cosine rule.

Consider the following triangle:

\begin{figure}[H]
	\includegraphics[width=\textwidth]{sinerule.png}
\end{figure}

The triangle ACX is right angled triangle, and the height $h$ can be calculated using the sine ratio formula:

\begin{equation}
h = b \times \sin(A)
\end{equation}

The triangle BXC is also a right angled triangle, height is calculated:

\begin{equation}
h = a \times \sin(B)
\end{equation}

Both of the equations are equal to the height of the triangle:

\begin{align}
a \times \sin(B) &= b \times \sin(A) \\
\frac{a}{\sin(A)} &= \frac{b}{\sin(B)}
\end{align}

Drawing a perpendicular line from A to BC will obtain similar results using $c$ and $\sin(C)$.

\textbf{The Sine Rule}

\begin{equation}
\frac{a}{\sin(A)} = \frac{b}{\sin(B)} = \frac{c}{\sin(C)}
\end{equation}

This rule is used is you have two angles, one length and want to find a missing length, or you have two lengths, one angle and want to find a missing angle.

\textbf{Example:}

Consider the triangle:

\begin{figure}[H]
	\includegraphics[width=\textwidth]{sineruleex.png}
\end{figure}

\begin{align}
A &= 180-72-51 = 57\\
\frac{x}{\sin(51)} &= \frac{12.4}{\sin(57)}\\
x &= 11.49033994 \approx 11.5\\
\frac{y}{\sin(72)} &= \frac{12.4}{\sin(57)}\\
y &= 14.06166051 \approx 14.1
\end{align}

\textbf{Questions:}
\begin{itemize}
	\item 28.1 - 2,4,6 (Page 511)
	\item 28.1 - 8,10,12 (Page 512)
\end{itemize}

\textbf{The ambiguous case of the sine rule}
You have to be careful when using the sine rule! An acute angle and obtuse angle can give the same result.

Consider a triangle:

\begin{figure}[H]
	\includegraphics[width=\textwidth]{sinecase.png}
\end{figure} 

Using the sine rule, the acute and obtuse angle made the side rotated through the arc with radius $7$ cm.

\begin{align}
\frac{7}{\sin(50\degree)} &= \frac{8}{\sin(C)}\\
\sin(C) &= 0.875479\\
C &= \arcsin(\frac{8 \times \sin(50)}{7}) = 61.10176\degree\\
\sin(180-61.10176) &= 0.875479372
\end{align}

\textbf{Questions:}
\begin{itemize}
	\item 2,4,6,8 Page(515.)
\end{itemize}

\textbf{The cosine rule}
The cosine rule is a version of Pythagoras's theorem and is used when calculating either at missing length or angle. If you need to calculate an angle, you need three lengths. If you need to calculate a length, you need an angle and two lengths.

\begin{align}
c^2 = a^2 + b^2 - 2ab\cos(C)\\
a^2 = b^2 + c^2 - 2ab\cos(A)\\
b^2 = a^2 + c^2 - 2ab\cos(B)
\end{align}

\begin{equation}
\cos(C) = \frac{a^2 + b^2 - c^2}{2ab}
\end{equation}

The other angles can be isolated from the other cosine rules.

\textbf{Questions:}
\begin{itemize}
	\item 3, 6, 8, 10 (Page 518)
	\item 2, 4, 8, 10 (Page 519)
\end{itemize} 

\section{Notes Thursday 16/1}

Consider the following triangle:

\begin{figure}[H]
	\includegraphics[width = \textwidth]{triArea.png}
\end{figure}

The area of the triangle is:

\begin{align}
\text{Area} &= \frac{1}{2} \times \text{base} \times \text{Height}\\
&= \frac{1}{2} \times AC \times BX\\
&= \frac{1}{2} \times b \times (a \times \sin(C))\\
\text{Area} &= \frac{1}{2}ab\sin(C)
\end{align}

Learn the formula in the three versions:

\begin{equation}
\text{Area} = \frac{1}{2}ab\sin(C)=\frac{1}{2}bc\sin(A)=\frac{1}{2}ac\sin(B)
\end{equation}

\textbf{Example:}
\begin{figure}[H]
	\includegraphics[width=\textwidth]{triAreaEx1.png}
\end{figure}


\begin{figure}[H]
	\includegraphics[width=\textwidth]{triAreaEx2.png}
\end{figure}

\begin{figure}[H]
	\includegraphics[width=\textwidth]{triAreaEx3.png}
\end{figure}

\textbf{Link between area and sector}

\begin{equation}
\text{Sector area} = \frac{\text{sector angle}}{360\degree} \times \pi \times r^2
\end{equation}


\begin{figure}[H]
	\includegraphics[width=\textwidth]{areaSectorLink.png}
\end{figure}

\textbf{Questions:}
\begin{itemize}
	\item 1, 2, 3, 4 
\end{itemize}

\section{Notes Friday 17/1}
\textbf{Important}
\begin{itemize}
	\item Break down the problem into two or more 2-D triangles.
	\item Draw a sketch, it will help you!
\end{itemize}

\begin{figure}[H]
	\includegraphics[width=\textwidth]{3dEx1.png}
\end{figure}

\begin{figure}[H]
	\includegraphics[width=\textwidth]{3dEx2.png}
\end{figure}

\begin{figure}[H]
	\includegraphics[width=\textwidth]{3dEx3.png}
\end{figure}
\end{document}