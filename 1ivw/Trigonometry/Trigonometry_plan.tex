\documentclass[10pt,a4paper]{article}
\usepackage[utf8x]{inputenc}
\usepackage[T1]{fontenc}
\usepackage{ucs}
\usepackage[english]{babel}
\usepackage{amsmath}
\usepackage{amsfonts}
\usepackage{amssymb}
\usepackage{graphicx}
\usepackage{float}
\usepackage{tikz}
\usepackage{siunitx}
\usepackage{gensymb}
\usepackage{venndiagram}
\graphicspath{{./figures/}}
\begin{document}

\title{Trigonometry}
\author{Nathan Hugh Barr}
\maketitle

Weeks: $2$

Dates: Monday $6/1$, Wednesday $8/1$, Thursday $9/1$, Tuesday $14/1$, Thursday $16/1$, Friday $17/1$

\section*{Monday 6/1}
\textbf{1st half module:}
\begin{itemize}
	\item Exam - Anu
\end{itemize}

\textbf{2nd half module:}
\begin{itemize}
	\item Math Activity
\end{itemize}

\section*{Wednesday 8/1}
\textbf{1st half module:}

\textbf{Trigonometrical functions}
\begin{itemize}
	\item Sine Ratio
	\item Cosine Ratio
	\item Tangent Ratio
\end{itemize}

\textbf{2nd half module:}

\textbf{Choosing the right trig function?}
\begin{itemize}
	\item SohCahToa
	\item How to use the SohCahToa triangles.
	\item Questions
\end{itemize}

\section*{Thursday 9/1}
\textbf{1st half module:}

\textbf{Recap of the trigonometrical functions}\\

\textbf{Finding an unknown angle}
\begin{itemize}
	\item Inverse functions
	\item Questions
\end{itemize}
\textbf{Angles of elevation and depression}
\begin{itemize}
	\item Elevation
	\item Depression
	\item Questions
\end{itemize}

\textbf{2nd half module:}

\textbf{Multi-stage problems}
	\begin{itemize}
		\item Triangles split into two right angle triangles.
		\item Pythagoras' Theorem
		\item Bearings
	\end{itemize} 

\section*{Tuesday 14/1}
\textbf{1st half module:}

\textbf{Sine Rule}
\begin{itemize}
	\item 
\end{itemize}
\textbf{Ambiguous case of the sine rule}\\

\textbf{2nd half module:}

\textbf{Cosine Rule}
\begin{itemize}
	\item 
\end{itemize}

\section*{Thursday 16/1}
\textbf{1st half module:}

\textbf{Recap of Right Angles}

\textbf{Recap of Sine rule and Cosine rule}\\

\textbf{2nd half module:}

\textbf{Area of a Triangle}
\begin{itemize}
	\item 
\end{itemize}

\section*{Friday 17/1}
\textbf{Trigonometry in 3-D:}


\section{Notes Wednesday 8/1}
\textbf{The Sine Ratio:}

Consider two triangles with similar angles. The ratio will be same regardless of the different lengths. The larger triangle has been enlarged by a scale factor. 

\begin{figure}[H]
	\includegraphics[width=\textwidth]{sinefig.png}
\end{figure}

In right angle triangles, the answer obtained by dividing the length opposite the angle by the hypotenuse is called the sine of that angle.

\begin{equation}
\sin(\theta) = \frac{\text{opposite}}{\text{hypotenuse}}
\end{equation}

Remember - Calculators set to degrees (DEG)

To calculate a missing length, Isolate the missing length from the sine formula.


\begin{figure}[H]
	\includegraphics[width=\textwidth]{basicTriangle.png}
\end{figure}
\begin{align}
\sin(\theta) = \frac{\text{opposite}}{\text{hypotenuse}}\\
\implies \text{opposite} = \sin(\theta) \times \text{hypotenuse}
\end{align}

\textbf{Example:}
\begin{figure}[H]
	\includegraphics[width=\textwidth]{sineEx.png}
\end{figure}

\begin{align}
\text{opposite} = \text{hypotenuse} \times \sin(\theta) \\
x = 4 \times \sin(38\degree)\\
x = 2.462645901\degree \approx 2.46\degree
\end{align}

Show that the hypotenuse can also be isolated.

\textbf{Cosine Ratio:}

Instead of trying to find the opposite length, the adjacent length can be found using the cosine ratio.

\begin{align}
\cos(\theta) = \frac{\text{adjacent}}{\text{hypotenuse}}\\
\implies \text{adjacent} = \text{hypotenuse} \times \cos(\theta)
\end{align}

\textbf{Tangent Ratio:}

The last ratio is defined as:

\begin{align}
\tan(\theta) = \frac{\text{opposite}}{\text{adjacent}}\\
\implies \text{opposite} = \text{adjacent} \times \tan(\theta)
\end{align}

\textbf{SohCahToa}
When solving questions the trigonometrical ratio rhyme can be used and triangles can be drawn to help isolate the missing length or angle(but more on that later).

\begin{figure}[H]
	\includegraphics[width=\textwidth]{sohcahtoa.png}
\end{figure} 
\textbf{How to use:} Cover the quantity you are looking for and the triangle will tell to either multiple or divide the quantities you were given in the question.

\textbf{Questions:}
\begin{itemize}
	\item 1 through to 8 (Page 317)
\end{itemize}

\section{Notes Thursday 9/1:}

\textbf{Finding an unknown angle:} 
\begin{itemize}
	\item Inverse - "Reverse" function
	\item Inverse Angle - Calculators -> shift, sin/cos/tan.
	\item \includegraphics[width=\textwidth]{example1.png}
	\item $\tan(\theta)=\frac{3}{4} \implies \theta = \arctan(\frac{3}{4})=36.86989765 \degree \approx 36.9\degree$ 
\end{itemize}

\textbf{Questions:}
\begin{itemize}
	\item 2,4,6,8,10,12 (Page 319)
\end{itemize}

\textbf{Angles of elevation and depression:}

Angle of elevation - Above the horizontal line

Angle of depression - Blow the horizontal line.

These problems are a combination of pythagoras' theorem, trigonometrical equations and alternate angles.

\textbf{Example:} A climber is sitting on the summit S of a mountain. He looks down and sees his camp C in the valley below. The direct distance from the summit to the camp is 2200 metres. The summit is at an altitude 420 meters higher than the camp.\\
a: Calculate the angle of elevation of the summit as seen from the camp. Give your answer to the nearest $0.1$\textdegree.

\begin{align}
\sin(\theta) = \frac{\text{opposite}}{\text{hypotenuse}}\\
\sin(\theta) = \frac{420}{2200}\\
\theta = \arcsin(\frac{420}{2200}) = 11.005\degree \approx 11\degree\\
\end{align}

b: Write down the angle of depression of the camp as seen from the summit. $=11\degree$

\textbf{Questions:}
\begin{itemize}
	\item 1,3,4 (Page. 325-326)
\end{itemize}

\textbf{Multi-stage problems:}

\end{document}