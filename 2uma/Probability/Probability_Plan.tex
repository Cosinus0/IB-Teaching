\documentclass[10pt,a4paper]{article}
\usepackage[utf8x]{inputenc}
\usepackage[T1]{fontenc}
\usepackage{ucs}
\usepackage[english]{babel}
\usepackage{amsmath}
\usepackage{amsfonts}
\usepackage{amssymb}
\usepackage{graphicx}
\usepackage{float}
\usepackage{tikz}
\usepackage{venndiagram}
\graphicspath{{./figures/}}
\begin{document}

\title{Probability}
\author{Nathan Hugh Barr}
\maketitle
	
Weeks: $2$

Dates: Monday $6/1$, Tuesday $7/1$, Monday $13/1$ (Cancelled because of Ess Work), Thursday $16/1$

\section*{Monday 6/1}
\textbf{1st half module}

\begin{itemize}
	\item What is probability?
	\begin{itemize}
		\item Represented numerically
		\item Determine probabilities
		\item Where is probability used?
	\end{itemize}
\end{itemize}

\begin{itemize}
	\item Experimental Probability
	\begin{itemize}
		\item Trials, Outcomes, Frequency, Relative frequency.
		\item Relative frequency $=$ Experimental probability(TOK 1)
		\item Example : Dice and Two way table
		\item Questions
	\end{itemize}
\end{itemize}

\textbf{2nd half module}

\begin{itemize}
	\item Sample Space
	\begin{itemize}
		\item Sample space, Events
		\item Set notation, subsets, Venn diagram
		\item complementary events
		\item Example: Venn diagram, 2D dimensional grid and Tree diagram
	\end{itemize}
\end{itemize}

\begin{itemize}
	\item Theoretical Probability
	\begin{itemize}
		\item Equally likely definition
		\item $P(A) = \frac{n(A)}{n(U)}$
		\item Complementary Events  $P(A) + P(A') = 1$
		\item Questions
	\end{itemize}
\end{itemize}

\section*{Tuesday 7/1}
\textbf{1st half module}
\begin{itemize}
	\item Addition law of probability
	\begin{itemize}
		\item Work through Investigation 4
		\item Definition $P(A \cup B) = P(A) + P(B) - P(A \cap B)$
		\item Mutually exculsive
		\item Examples:
		\item Questions
	\end{itemize}
\end{itemize}

\textbf{2nd half module}
\begin{itemize}
	\item Independent events
	\begin{itemize}
		\item Work through Investigation 5
		\item Definition $P(A \cap B) = P(A) \times P(B)$
		\item Examples:
		\item Questions
	\end{itemize}
\end{itemize}

\section*{Thursday 16/1}
\textbf{1st half module}
\begin{itemize}
	\item Dependent events
	\begin{itemize}
		\item Example:
		\item Definition $P(A \cap B) = P(A) \times P(B \mid A)$
		\item Experiments and Replacement
		\item Questions
	\end{itemize}
\end{itemize}

\textbf{2nd half module}
\begin{itemize}
	\item Conditional probability
	\begin{itemize}
		\item Example:
		\item Definition $P(A \mid B) = \frac{n(A \cap B)}{n(B)}$
		\item Equally likely $\implies$ conditional probability $P(A \mid B) = \frac{P(A \cap B)}{P(B)}$
		\item Questions
	\end{itemize}
\end{itemize}


\section{Notes to Monday 6/1}

\textbf{Probability:} 
In the real world we think of probability as a \textbf{chance or likelihood} of some event happening. We assign a number between $0$ and $1$ to the chance of a event occurring and we call this number the probability. 

\begin{figure}[H]
	\includegraphics[width=\textwidth]{fig1}
\end{figure}

We can determine probabilities based on: \textbf{the results of an experiment} and/or \textbf{what we theoretically expect to happen}.

The theory of probability is used in a wide range of fields:
\begin{itemize}
	\item Biology
	\item Economics
	\item Politics
	\item Sport
	\item Quality control
	\item Production planning
	\item Physics - Quantum mechanics and Statistical mechanics
\end{itemize}

\textbf{Experimental Probability}:
When performing an experiment that involves chance the following information is needed.
\textbf{Number of trials:} is the total number of times the experiment is performed.
\textbf{Outcomes:} are the different results possible for one trial of the experiment.
\textbf{Frequency of an outcome:} is the number of times the outcome is observed.
\textbf{Relative frequency of an outcome:} is the frequency of the outcome expressed as a fraction or percentage of the total number of trials. $RF = \frac{Frequency}{Number of trials}$\\


\textbf{Example: Dice probability experiment}
\begin{itemize}
	\item What is the theoretical probability of each face?
	\item How many trials will you do?
	\item What outcomes are possible? (Express in set notation)
	\item Express the frequency of an outcome in your experiment.
	\item Calculate the relative frequency of an outcome in your experiment.
	\item (Look at individual experiments vs. the sum of the trials) 
\end{itemize}

In experiments, the relative frequency is the best estimate of the probability of that event occurring, \textbf{Relative frequency $=$ Experimental probability}.

\textbf{Example: Two way table}

\begin{tabular}{|c|c|c|c|}
	\hline 
	& Adult & Child & Total \\ 
	\hline 
	Season ticket holder & 1824 & 779 & 2603 \\ 
	\hline 
	Not a season ticket holder & 3247 & 1660 & 4907 \\ 
	\hline 
	Total & 5071 & 2439 & 7510 \\ 
	\hline 
\end{tabular}
\begin{equation}
b_i: P(\text{A child}) = \frac{779+1660}{7510} = 0.32 \implies 32\%
\end{equation}

\begin{equation}
b_{ii}: P(\text{Not a season ticket holder}) = \frac{3247+1660}{7510} = 0.65 \implies 65\% 
\end{equation}

\begin{equation}
b_{iii}: P(\text{An adult season ticket holder}) = \frac{1824}{7510} = 0.24 \implies 24\%
\end{equation}

\textbf{Questions:}
\begin{itemize}
	\item 10A - 1, 2, 3 (Page 243)
	\item 10A - 5 (Page 244)
	\item 10B - 3, 4 (Page 247)\\ 
\end{itemize}

\textbf{Sample space, events and complementary events}\\
\begin{itemize}
	\item The \textbf{sample space} $U$ is the set of all possible outcomes of an experiment. 
	\item An \textbf{event} is a set of outcomes in the sample space that have a particular property.
	\item The sample space is the \textbf{universal set} $U$.
	\item The outcomes are the \textbf{elements} of the sample space.
	\item Events are \textbf{subsets} of the sample space.
	\item We use set notation and Venn diagrams to solve probability problems.
\end{itemize}

\textbf{Complementary events:} Two events are \textbf{complementary} if exactly one of the events must occur. If A is an event, then A' is the complementary event of A, or "not A".

\textbf{Example: Venn diagram Q 10C.3}
\begin{align}
U &= \{1, 2, 3, 4, 5, 6, 7, 8, 9, 10, 11, 12, 13, 14, 15, 16\}\\
A &= \{4, 8, 12, 16 \}
\end{align}
\begin{venndiagram2sets}
\end{venndiagram2sets}	
\end{document}