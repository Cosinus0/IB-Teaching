\documentclass[10pt,a4paper]{article}
\usepackage[utf8x]{inputenc}
\usepackage[T1]{fontenc}
\usepackage{ucs}
\usepackage[english]{babel}
\usepackage{amsmath}
\usepackage{amsfonts}
\usepackage{amssymb}
\usepackage{graphicx}
\usepackage{float}
\usepackage{tikz}
\usepackage{venndiagram}
\graphicspath{{./figures/}}
\begin{document}

\title{Probability}
\author{Nathan Hugh Barr}
\maketitle
	
Weeks: $2$

Dates: Monday $6/1$, Tuesday $7/1$, Monday $13/1$ (Cancelled because of Ess Work), Thursday $16/1$

\section*{Monday 6/1}
\textbf{1st half module}

\begin{itemize}
	\item What is probability?
	\begin{itemize}
		\item Represented numerically
		\item Determine probabilities
		\item Where is probability used?
	\end{itemize}
\end{itemize}

\begin{itemize}
	\item Experimental Probability
	\begin{itemize}
		\item Trials, Outcomes, Frequency, Relative frequency.
		\item Relative frequency $=$ Experimental probability(TOK 1)
		\item Example : Dice and Two way table
		\item Questions
	\end{itemize}
\end{itemize}

\textbf{2nd half module}

\begin{itemize}
	\item Sample Space
	\begin{itemize}
		\item Sample space, Events
		\item Set notation, subsets, Venn diagram
		\item complementary events
		\item Example: Venn diagram, 2D dimensional grid and Tree diagram
	\end{itemize}
\end{itemize}

\begin{itemize}
	\item Theoretical Probability
	\begin{itemize}
		\item Equally likely definition
		\item $P(A) = \frac{n(A)}{n(U)}$
		\item Complementary Events  $P(A) + P(A') = 1$
		\item Questions
	\end{itemize}
\end{itemize}

\section*{Tuesday 7/1}
\textbf{1st half module}
\begin{itemize}
	\item Addition law of probability
	\begin{itemize}
		\item Work through Investigation 4
		\item Definition $P(A \cup B) = P(A) + P(B) - P(A \cap B)$
		\item Mutually exclusive
		\item Examples:
		\item Questions
	\end{itemize}
\end{itemize}

\textbf{2nd half module}
\begin{itemize}
	\item Independent events
	\begin{itemize}
		\item Work through Investigation 5
		\item Definition $P(A \cap B) = P(A) \times P(B)$
		\item Questions
	\end{itemize}
\end{itemize}

\section*{Thursday 16/1}
\textbf{1st half module}
\begin{itemize}
	\item Dependent events
	\begin{itemize}
		\item Example:
		\item Definition $P(A \cap B) = P(A) \times P(B \mid A)$
		\item Experiments and Replacement
		\item Questions
	\end{itemize}
\end{itemize}

\textbf{2nd half module}
\begin{itemize}
	\item Conditional probability
	\begin{itemize}
		\item Example:
		\item Definition $P(A \mid B) = \frac{n(A \cap B)}{n(B)}$
		\item Equally likely $\implies$ conditional probability $P(A \mid B) = \frac{P(A \cap B)}{P(B)}$
		\item Questions
	\end{itemize}
\end{itemize}


\section{Notes to Monday 6/1}

\textbf{Probability:} 
In the real world we think of probability as a \textbf{chance or likelihood} of some event happening. We assign a number between $0$ and $1$ to the chance of a event occurring and we call this number the probability. 

\begin{figure}[H]
	\includegraphics[width=\textwidth]{fig1}
\end{figure}

We can determine probabilities based on: \textbf{the results of an experiment} and/or \textbf{what we theoretically expect to happen}.

The theory of probability is used in a wide range of fields:
\begin{itemize}
	\item Biology
	\item Economics
	\item Politics
	\item Sport
	\item Quality control
	\item Production planning
	\item Physics - Quantum mechanics and Statistical mechanics
\end{itemize}

\textbf{Experimental Probability}:
When performing an experiment that involves chance the following information is needed.

\begin{itemize}
	\item \textbf{Number of trials:} is the total number of times the experiment is performed.
	\item \textbf{Outcomes:} are the different results possible for one trial of the experiment. 
	\item \textbf{Frequency of an outcome:} is the number of times the outcome is observed. 
	\item \textbf{Relative frequency of an outcome:} is the frequency of the outcome expressed as a fraction or percentage of the total number of trials. $RF = \frac{Frequency}{Number of trials}$\\ 
\end{itemize}


\textbf{Example: Dice probability experiment}
\begin{itemize}
	\item What is the theoretical probability of each face?
	\item How many trials will you do?
	\item What outcomes are possible? (Express in set notation)
	\item Express the frequency of an outcome in your experiment.
	\item Calculate the relative frequency of an outcome in your experiment.
	\item (Look at individual experiments vs. the sum of the trials) 
\end{itemize}

In experiments, the relative frequency is the best estimate of the probability of that event occurring, \textbf{Relative frequency $=$ Experimental probability}.\\

\textbf{Example: Two way table}

\begin{tabular}{|c|c|c|c|}
	\hline 
	& Adult & Child & Total \\ 
	\hline 
	Season ticket holder & 1824 & 779 & 2603 \\ 
	\hline 
	Not a season ticket holder & 3247 & 1660 & 4907 \\ 
	\hline 
	Total & 5071 & 2439 & 7510 \\ 
	\hline 
\end{tabular}
\begin{equation}
b_i: P(\text{A child}) = \frac{779+1660}{7510} = 0.32 \implies 32\%
\end{equation}

\begin{equation}
b_{ii}: P(\text{Not a season ticket holder}) = \frac{3247+1660}{7510} = 0.65 \implies 65\% 
\end{equation}

\begin{equation}
b_{iii}: P(\text{An adult season ticket holder}) = \frac{1824}{7510} = 0.24 \implies 24\%
\end{equation}

\textbf{Questions:}
\begin{itemize}
	\item 10A - 1, 2, 3 (Page 243)
	\item 10A - 5 (Page 244)
	\item 10B - 3, 4 (Page 247)\\ 
\end{itemize}

\textbf{Sample space, events and complementary events}\\
\begin{itemize}
	\item The \textbf{sample space} $U$ is the set of all possible outcomes of an experiment. 
	\item An \textbf{event} is a set of outcomes in the sample space that have a particular property.
	\item The sample space is the \textbf{universal set} $U$.
	\item The outcomes are the \textbf{elements} of the sample space.
	\item Events are \textbf{subsets} of the sample space.
	\item We use set notation and Venn diagrams to solve probability problems.
\end{itemize}

\textbf{Complementary events:} Two events are \textbf{complementary} if exactly one of the events must occur. If A is an event, then A' is the complementary event of A, or "not A".

\textbf{Example: Venn diagram Q 10C.3}
\begin{align}
U &= \{1, 2, 3, 4, 5, 6, 7, 8, 9, 10, 11, 12, 13, 14, 15, 16\}\\
A &= \{4, 8, 12, 16 \}\\
B &= \{1, 4, 9, 16\}
\end{align}

\begin{venndiagram2sets}[tikzoptions={scale=2.8}, labelOnlyA={8  12}, labelOnlyB={1  9}, labelAB={4  16}, labelNotAB={\, \, \, \, \; \; 2 3 5 6 7 10 11 13 14 15}]
\end{venndiagram2sets}

\textbf{Example: 2D grid and Tree Diagram}
\begin{figure}[H]
	\includegraphics[width=\textwidth]{fig2.png}
\end{figure}

\textbf{Theoretical probability:}
\begin{itemize}
	\item \textbf{Equally Likely:} If a sample space has n outcomes which are equally likely to occur when the experiment is performed once, then each outcome has probability $\frac{1}{n}$ of occuring.
	\begin{itemize}
		\item Example: Dice - 6 sides, each side is equally likely be rolled. Thus each side has a probability $\frac{1}{6}$ of occuring.
	\end{itemize}
	\item When the outcome of an experiment are equally likely, the probability that an event A occurs is:
	\begin{itemize}
		\item $P(A) = \frac{\text{number of outcomes corresponding to A}}{\text{number of outcomes in the sample space}}= \frac{n(A)}{n(U)}.$
		\item Example: Dice - Probability of rolling either a 1 or a 5, Event $A = \text{rolling either a 1 or a 5}$, $n(A)=2$ and $n(U)=6$.
		\item What is the complementary event $A'$?
		\item $P(A) + P(A') = 1$, A good sanity check!
	\end{itemize}    
\end{itemize}
\textbf{Questions:}
\begin{itemize}
	\item 10C - 4, 5 (Page 249)
	\item 10D - 1, 7 (Page 252)
	\item 10D - 8, 10 (Page 253)
	\item 10D - 12(Use the example 9 above) (Page 254)
	\item 10D - 19 (Page 255) 
\end{itemize}	

\section{Notes to Tuesday 7/1}
\textbf{Addition law of probability}
\begin{itemize}
	\item \textbf{Compound events:} More than one event in our sample space. This could be two or more processes in our experiment.
	\begin{itemize}
		\item Two events A and B.
		\item Both A and B occurs written as $A \cap B$, reads as "A intersection B".
		\item A or B or Both occurs written as $A \cup B$, reads as "A union B".
	\end{itemize}
	\item How are multiple probabilities added together?
	\item Investigation 4 - 2 Event Venn diagram (Page 258)
	\begin{itemize}
		\item Suppose $U=\{x\mid \text{positive integers less than 100}\}$.
		\item Let $A=\{\text{multiplies of 7 in U}\}$ and let $B=\{\text{multiples of 5 in U}\}$.
		\item How many elements in:
		\begin{itemize}
			\item $U$ (99)
			\item $A$ (14)
			\item $B$ (19)
			\item $A \cap B$ (2)
			\item $A \cup B$ (31)
			\item \begin{align}
				    n(A \cup B) = n(A) + n(B) - n(A \cap B) \\
				    31 = 14 + 19 - 2	
				  \end{align}
		\end{itemize}
		\item The probability of the events:
		\begin{itemize}
 			\item $P(A) = \frac{a + b}{a+b+c+d}$.
 			\item $P(B)$, $P(A\cap B)$, $P(A \cup B)$ and $P(A)+P(B)-P(A\cap B)$.
 			\item What is the connection between $P(A \cup B)$ and $P(A)+P(B)-P(A\cap B)$.
		\end{itemize} 
	\end{itemize} 
	\item \textbf{Addition law of probability:} $P(A \cup B)=P(A) + P(B) - P(A\cap B)$
	\item \textbf{Mutually exclusive:} A and B are \textbf{disjointed} events $P(A\cap B) = 0$, the addition law becomes $P(A \cup B)=P(A)+P(B)$.  
\end{itemize}

\textbf{Example: (Page 259)}
\begin{figure}[H]
	\includegraphics[width=\textwidth]{fig3.png}
\end{figure}

\textbf{Questions:}
\begin{itemize}
	\item 10E - 3, 4 (Page 259)
	\item 10E - 5, 6, 7 (Page 260)
\end{itemize}

\textbf{Independent Events}
\begin{itemize}
	\item Two events are \textbf{independent} if the occurrence of each event does not affect the occurrence of the other.
	\item How can we calculate $P(A \cap B)$ for two independent events?
	\begin{itemize}
		\item Suppose a coin is tossed and a die is cast at the same time.
		\item Does the outcome of the coin toss affect the die roll?
		\item 2-Dimensional grid
		\item \begin{tabular}{|c|c|c|c|c|}
			\hline 
			$A$ & $B$ & $P(A)$ & $P(B)$ & $P(A \cap B)$ \\ 
			\hline 
			Head & 4 & $\frac{1}{2}$ & $\frac{1}{6}$ & $\frac{1}{12}$  \\ 
			\hline 
			Head &  Odd number & $\frac{1}{2}$ & $\frac{1}{2}$  & $\frac{1}{4}$  \\ 
			\hline 
			Tail & Number greater than 1 & $\frac{1}{2}$  & $\frac{5}{6}$  & $\frac{5}{12}$  \\ 
			\hline 
			Tail & Number less than 3 & $\frac{1}{2}$  & $\frac{1}{3}$  & $\frac{1}{6}$  \\ 
			\hline 
		\end{tabular}
	\item Selecting balls from two boxes. 1st box - BBGG and 2nd box RRRW.
	\item Does the outcome from one box affect the other?
	\item 2-Dimensional Grid
	\item \begin{tabular}{|c|c|c|c|c|}
		\hline 
	$A$	& $B$  & $P(A)$  & $P(B)$  & $P(A \cap B)$  \\ 
		\hline 
	Green from box X & Red from box Y & $\frac{1}{2}$  & $\frac{3}{4}$  & $\frac{3}{8}$  \\ 
		\hline 
	Green from box X & Red from box Y & $\frac{1}{2}$  & $\frac{3}{4}$  & $\frac{3}{8}$  \\ 
		\hline 
	Blue from box X & Red from box Y & $\frac{1}{2}$  & $\frac{3}{4}$  & $\frac{3}{8}$  \\ 
		\hline 
	Blue from box X	& White from box Y & $\frac{1}{2}$  & $\frac{1}{4}$  & $\frac{1}{8}$  \\ 
		\hline 
	\end{tabular}   
	\end{itemize}
		\item For independent events A and B, what is the connection between $P(A \cap B)$, $P(A)$ and $P(B)$?
\end{itemize}

\textbf{Independent Events:} If A and B are independent events, then $P(A \cap B) = P(A) \times P(B)$.\\

\textbf{Multiple Independent Events:} If A, B and C are independent events, then $P(A \cap B \cap C) = P(A) \times P(B) \times P(C)$. 

\textbf{Questions:}
\begin{itemize}
	\item 10F - 1, 2, 3 (Page 261)
	\item 10F - 5, 6 (Page 262)
	\item 10F - 8, 9, 11 (Page 263)  
\end{itemize}

\section{Notes to Thursday 16/1}
\textbf{Dependent events:}
\begin{itemize}
	\item Sampling example:
	\begin{itemize}
		\item 5 Red and 3 blue tickets in a box. One ticket is drawn, colour noted and not put back, and a second ticket is drawn.
		\item Whats the chance that it is red?
		\item First ticket is red, $P(\text{second is red})= \frac{4}{7}$
		\item First ticket is blue, $P(\text{second is red})= \frac{5}{7}$
		\item The probability of the second ticket being red depends on the what the first ticket was!
	\end{itemize}
	\item Definition:
	\item Two or more events are \textbf{Dependent} if the occurrence of one of the events does affect the occurence of the other events. 
	\item Events are \textbf{dependent} is they are \textbf{not independent}.
	\item If A and B are dependent events then $P(A \cap B) = P(A) \times P(\text{B given that A has occurred}).$
	\item Sampling:
	\begin{itemize}
		\item \textbf{Without Replacement} we have dependent events.
		\item \textbf{Without Replacement} we have independent events.
	\end{itemize} 
\end{itemize}


\textbf{Questions:}
\begin{itemize}
	\item 10G - 1,2 (Page 265)
	\item 10G - 5,6 (Page 266)
	\item 10G - 7, 10 (Page 267)
	\item 10G - 15 (Page 268)
\end{itemize}

\textbf{Conditional probability:}
\begin{itemize}
	\item \textbf{Example:} The probability that a randomly chosen student who studies french, also studies italian? 
	\item \begin{venndiagram2sets}[labelA={I},labelB={F},labelOnlyA={(14)}, labelOnlyB={(7)}, labelAB={(8)}, labelNotAB={(1)}]
	\end{venndiagram2sets}
	\item $P(\text{I given that F has occured})=\frac{8}{15}=\frac{\text{number students that studies Italian and French.}}{\text{number of students who study French.}}$
\end{itemize}
\textbf{Conditional probability definition:}For events A and B, we use the notation "A$\mid$B" to represent the event "A given that B has occured".
\begin{equation}
P(A \mid B)=\frac{n(A \cap B)}{n(B)}
\end{equation}

If the outcomes of the events are equally likely,

\begin{equation}
\frac{n(A \cap B)}{n(B)} = \frac{\frac{n(A \cap B)}{n(U)}}{\frac{n(B)}{n(U)}} = \frac{P(A \cap B)}{P(B)}
\end{equation}

\textbf{Conditional probability formula:}
\begin{equation}
P(A \mid B)=\frac{P(A \cap B)}{P(B)}
\end{equation}

\textbf{Questions:}
\begin{itemize}
	\item 10H - 1, 2, 4 (Page 269)
	\item 10H - 7 (Page 270)
	\item 10H - 9 (Page 271)
\end{itemize}
\end{document}